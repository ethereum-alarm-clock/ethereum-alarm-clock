\documentclass{article}
\usepackage[utf8]{inputenc}
\usepackage{mathtools}
\title{The on-chain claiming mechanism economics}
\author{Piotr Kosiński\\ChronoLogic}
\begin{document}
\maketitle
\section{Introduction}
This article describes the economic model for on-chain claiming mechanism used by Ethereum Alarm Clock protocol. This mechanism has been implemented to improve the economics of running the TimeNode (executing agent), especially given the network of TimeNodes $n$.
\section{Claiming mechanism}
Claiming mechanism can be described as follows:
For any transaction $Tx$ that has been deployed to the network and is expected to be executed by 1 of $n$ nodes in the network. The process of execution is divided into two steps:
\textbf{claiming} and \textbf{execution}. Claiming is a process of reserving the transaction for further execution.

\subsection{Claiming}
\begin{itemize}
\item can happen before execution
\item every node has the same chance to successfully claim $Tx$
\item claiming requires a $Deposit$ to be locked by claimant
\item $Deposit$ is lost by claimant when execution won't happen within \textit{exclusive execution window}
\item every node can fail on claiming when $Tx$ was already claimed
\item claiming requires sending transaction that has cost described as $C_{c}$ when success and $C_{f}$ when failure
\item payment modifier $P_{mod}(t)=\begin{cases}
0 & \quad \text{at beginning of claiming window}\\
1 & \quad \text{at end of claiming window}
\end{cases}$ 
\item claiming is optional
\end{itemize}

\subsection{Execution}
\begin{itemize}
\item successful execution has reward described as $TimeBounty$ 
\item execution cost is reimbursed by the scheduler when successful
\item execution cost has cost described as $C_{e}$ when not successful
\item $Deposit$ locked by the claimant can be acquired by a node when the claimant failed to execute
\end{itemize}

\subsection{Expected payout definition}
Let's define the expected payout for node as
\[
P=P_{s}+P_{f}+P_{nf}
\]
where
\\

$P_{s}$ is expected payout after successful claiming and execution
\\

$P_{f}$ is expected payout after successful claiming and missed execution
\\

$P_{nf}$ is expected payout after other node losing the deposit
\\

\subsubsection{Network with $n=1$ nodes}
For network of nodes with $n=1$ we can define expected payouts as:
\[
P_{s}=P_{mod} \times TimeBounty-C_{C} 
\]
\[
P_{f}=-C_{C}-Deposit
\]
\[
P_{nf}=TimeBounty-C_{C}
\]
\subsubsection{Network with $n>1$ nodes}
For that case the expected reward will be $\frac{TimeBounty}{n}$ assuming that probability of successful claiming is equal for all nodes. Also in case of failing transaction node will pay the $C_{Tx}$
\[
P_{s}=\frac{P_{mod} \times TimeBounty-C_{C}}{n} - (n-1) \times C_{Tx}
\]
\[
P_{f}=-C_{C}-Deposit
\]
\[
P_{nf}=\frac{TimeBounty-C_{C}}{n} - (n-1) \times C_{Tx}
\]

In order to improve the cost of failing transactions let's introduce a mechanism $X$ that prevents sending transaction that expect to fail, the accuracy of mechanism $X$ is defined as $A_{X} \in [0;1]$

\[
P_{s}(A_{X})=\frac{P_{mod} \times TimeBounty-C_{C}}{n} - (1-A_{X}) \times (n-1) \times C_{Tx}
\]
\[
P_{f}=-C_{C}-Deposit
\]
\[
P_{nf}(A_{X})=\frac{TimeBounty-C_{C}}{n} - (1-A_{X}) \times (n-1) \times C_{Tx}
\]

The last part is to introduce $P_{ld} \in [0;1]$ as probability of TimeNode loosing the $Deposit$

\[
P(A_{X}, P_{ld})=P_{s}(A_{X}) \times (1-P_{ld}) + (P_{f}(A_{X})+P_{nf}(A_{X})) \times P_{ld}
\]
\section{Simulations}
\end{document}
